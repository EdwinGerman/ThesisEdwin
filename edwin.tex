%% -*- coding: utf-8; -*-

\documentclass[
  phd,
  %% master
  %% brazilian
  american
]{ThesisPUC}


%%%
%%% Additional Packages
%%%

  %% \usepackage[brazilian]{babel}      %% in ThesisPUC.cls
  %% \usepackage[utf8]{inputenc}        %% .
  %% \usepackage[T1]{fontenc}           %% .
  %% \usepackage{lmodern}               %% .
  %% \usepackage[pdftex]{graphicx}	%% .

  \usepackage{tabularx}
  \usepackage{multirow}
  \usepackage{multicol}
  \usepackage{colortbl}
  \usepackage[%
    dvipsnames,
    svgnames,
    x11names,
    fixpdftex
  ]{xcolor}
  \usepackage{numprint}
  \usepackage{textcomp}
  \usepackage{booktabs}
  \usepackage{amsmath}
  \usepackage{enumitem}
  \usepackage{amssymb}
  \usepackage{textcomp}
% \usepackage{etoolbox}
 %% numprint 
\npthousandsep{.}
\npdecimalsign{,}



%% ThesisPUC option
%% \tablesmode{figtab} %% [nada, fig, tab ou figtab]


%%%
%%% Counters
%%%

%% uncomment and change for other depth values
%% \setcounter{tocdepth}{3}
%% \setcounter{lofdepth}{3}
%% \setcounter{lotdepth}{3}
%% \setcounter{secnumdepth}{3}


%%%
%%% New commands and other global definitions
%%%

\input{defs}


%%%
%%% Misc.
%%%

\usecolour{true}


%%%
%%% Titulos
%%%

\author{Edwin Germán Maldonado Távara}
\authorR{Maldonado Távara, Edwin Germán}

\advisor{Marley Maria B. R. Vellasco}{Prof.}
\advisorR{Vellasco, Marley}

\coadvisor{Bruno A. C. Horta}{Dr.}
\coadvisorR{Horta, Bruno}
\coadvisorInst{Universidade Federal de Rio de Janeiro}{LABBMOL/UFRJ}

\coadvisorII{Fábio Lima Custódio}{Dr.}
\coadvisorIIR{Custodio, Fabio}
\coadvisorIIInst{Laboratorio Nacional de Computacão Científica}{LNCC}

%% \title{Desenvolvimento de um sistema de microscopia digital para
%%  classificação automática de tipos de hematita em minério de ferro}

\title{Investigação da impôrtancia relativa de features usados na
predição da estrutura de proteinas: 
Uma abordagem usando Aprendizado de Máquina}

\titleuk{Investigation of the relative importance of
features used in protein structure
prediction: a machine learning approach.}

%% \subtitulo{Aqui vai o subtitulo caso precise}

\day{30}
\month{August}
\year{2017}

\city{Rio de Janeiro}
\CDD{620.11}
\department{Engenharia Elétrica}
\program{Engenharia Elétrica}
\school{Centro Técnico Científico}
\university{Pontifícia Universidade Católica do Rio de Janeiro}
\uni{PUC-Rio}


%%%
%%% Jury
%%%

\jury{%
  \jurymember{André Vargas Abs.}{Prof.}
    {Universidade do Estado do Rio de Janeiro}{UERJ}
  \jurymember{Karla Figeirredo}{Profa.}
    {Universidade do Estado do Rio de Janeiro}{UERJ}
  \jurymember{Laurent Emmanuel Dardenne}{Prof.}
    {Laboratório Nacional de Computação Científica}{LNCC}
  \jurymember{Marcos Henrique de Pinho Maurício}{Dr.}
    {Departamento de Engenharia de Materiais}{PUC-Rio}
  \schoolhead{José Eugenio Leal}{Prof.}
}


%%%
%%% Resume
%%%

\resume{%
 Graduated in Systems Engineering at UNT (National Uni-
 versity of Trujillo - La Libertad, Perú) in 2002. Did his mas-
 ter degree at PUC-Rio, specializing in the application of op-
 timization and machine learning methods in Bioinformatics.
 Dedicated to full-time research at PUC-Rio, Brazil.}


%%%
%%% Acknowledgment
%%%

\acknowledgment{%
  \noindent I would like to first thank my advisor ...
  \bigskip

  \noindent Then I wish to thank ...
}


%%%
%%% Catalog prekeywords
%%%

\catalogprekeywords{%
  \catalogprekey{Engenharia El\'etrica}%
  \catalogprekey{Inteligência Computacional Aplicada}%
}


%%%
%%% Keywords
%%%

\keywords{%
  \key{Predição de Estrutura de Prote\'{i}nas;}
  \key{Avaliação da Qaulidade de Proteínas;}
  \key{Seleção de descriptores;}
  \key{Importância relativa de descriptores;}
  \key{Aprendizado de Máquina;}
  \key{Redes Neurais;}
  \key{Algoritmos Genéticos.}
}

\keywordsuk{%

   \key{Protein Structure Prediction;}
   \key{Protein Quality Assessment;}
   \key{Feature Selection;}
   \key{Features Relative Importance;}
   \key{Machine Learning;}
   \key{Neural Networks;}
   \key{Genetic Algorithm.}
 
}


%%%
%%% Abstract
%%%

\abstract{%
 As proteinas são importantes pois estas determinam as funções das células vivas.
 Estas funções estão diretamente relacionadas com sua estrutura terciaria. Para determinar
 a estrura terciaria de proteinas tem sido desenvolvidos muitos metódos experimentais.
 Sin embargo, estes metódos são costosos e demorados. Por isso, métodos computacionais 
 foram desenvolvidos para a predição da estrutura terciária da proteína.
 Esses métodos geram um conjunto de vários modelos candidatos conhecidos como decoys. 
 Neste conjunto, identificar o modelo ou o subconjunto de modelos mais próximos da estrutura nativa da proteína, 
 é uma tarefa desafiadora. Para realizar esta tarefa, foram propostos dois tipos de métodos.
 O primeiro grupo de métodos utilizam a estrutura nativa para avaliar a qualidade 
 dos decoys (RMSD, GDT-TS, TM-Score, MaxSub). 
 Na ausência da estrutura nativa, foram propostas  métodos baseados em funções de pontuação 
  (funções de potencial, funções de potencial estatístico, funções baseadas em consenso,
 algoritmos de aprendizado de máquina).
 Os métodos de aprendizado de máquina, avaliam a qualidade dos modelos decoys  usando um subconjunto de descriptores. 
 A vantagem desses métodos é a caracteristica de identificar relações ocultas entre os descriptores selecionados, 
 esta tarefa é difícil de alcançar com os outros tipos de métodos. Por outro lado, 
 estes métodos tem a desvantagem de realizar uma seleção não automatica de descriptores, 
 isto pode ser um fator limitante na avaliação da qualidade dos decoys.
 Dadas estas desvantagens, este trabalho propõe um modelo para a seleção automática de descriptores, 
 esta seleção é realizada usando diferentes tipos de descriptores. Além disso, este trabalho introduz um novo 
 método  para o cálculo da importancia relativa dos features chamados SWToFIC.
 Finalmente, o modelo fornece a predição do GDT-TS score para a avaliação da qualidade dos decoys de proteína.}

\abstractuk{%
 Proteins are important because they determine different functions in living cells. These functions, are directly related to its three-dimensional structure. To determine the protein three-dimensional structure have been developed several experimental methods. However, these methods are costly and time-consuming. Therefore, computational methods have been developed for protein tertiary structure prediction. These methods generate a set of several candidate’s models known as decoys. In this set, identify the model or subset of models that are more close to the protein native structure is a challenging task. To perform this task two types of methods have been proposed. The first types of methods use the native structure to evaluate the decoys quality (RMSD,GDT-TS,TM-Score,MaxSub). In absence of the native structure, methods based scoring functions have been proposed (physics-based potential functions, statistical potential functions, consensus-based functions, machine learning algorithms). Machine learning methods, evaluate the models quality using a subset of features. The advantage of these methods is their power to identify hidden relationships between the selected features, this task is difficult to achieve with the other methods. On the other hand, the main disadvantage is that these methods perform a non-automatic feature selection, this could be a factor that limits the quality assessment of the decoys models. Given these drawbacks, this work proposes a model for the automated feature selection, this selection is performed through different types of features. Additionally, this work introduces a new wrapper method to calculates the relative feature importance called (SWtoFIC). Finally, the model provides the GDT-TS score prediction for assessing the quality of the protein decoys.}


%%%
%%% Dedication
%%%

\dedication{%
  To my parents, for their support\\
and encouragement.
}

%%%
%%% Epigraph
%%%

\epigraph{%
  Like all other arts, the Science of Deduction and Analysis
  is one which can only be acquired by long and patient study,
  nor is life enough to allow any mortal to attain the highest possible perfection in it. Before turning to those moral and mental aspects of the matter which present the greatest difficulties, let the inquirer begin by mastering more elementary problems. 
}
\epigraphauthor{Sir Arthur Conan Doyle}
\epigraphbook{Sherlock Holmes, A Study in Scarlet}


%%%
%%% 
%%%

\begin{document}
  \input{abrevs}
  % -*- coding: utf-8; -*-

\chapter{Introduction}

\section{Motivation}

In nature, there are 20 different natural amino acids, each of them exhibiting different pysicochemical properties (e.g. size, charge, hydrophobicity)\cite{1}\cite{2}. Depending on their polarity, amino acids  vary in their hydrophilic or hydrophobic character, which is crucial for the formation of more complex structures. A peptide is a molecule composed of two or more amino acid residues linked through a covalent chemical bond called the peptide bond. This peptide bond is formed when the carboxyl group of a residue reacts with the amino group of another residue,releasing one water molecule. Large peptides are generally called polypeptides or proteins\cite{3}\cite{4}. Proteins perform a variety of functions in living organisms such as catalysts of specific reactions, structural components of cell membranes, antibodies, carry oxygen in the blood and are part of chromosome material. On the other hand, proteins control the regulation and reproduction of living things and are thus essential for life.%that are vital for life.Furthermore, proteins are important components of cell %membranes, function as antibodies,carry oxygen in the blood and are part of chromosome %material. Proteins control the regulation and reproduction of living things and are %thus essential for life.

The structure of proteins can be divided into four levels\cite{1}\cite{2}  (a) primary structure, (b) secondary structure, (c) tertiary structure and (d) Quaternary structure. The primary structure corresponds to the sequence (linear order) of the amino acids forming a given protein\cite{1}\cite{2}\cite{4}\cite{5}. The beginning of the primary structure corresponds to its N-terminal and the end of its primary structure is the C-terminal region. The secondary structure is defined by characteristic structural arrangements that are the consequence of hydrogen-bond patterns involving the backbone atoms (the amide chemical function that forms the main chain of the peptide)\cite{2}. The most common regular secondary structures are $\alpha$-helices and $\beta$-sheets\cite{6}, which are usually highly stable and constitute key elements in the three-dimensional (3D) structure of proteins. The tertiary structure of a protein is represented by the distribution of secondary structures in a 3D space. 
\newpage
The three-dimensional shape assumed by a protein is also called its native, functional or folded structure. The native structure is the lowest free-energy conformation under physiological condition, which can be viewed as a consequence of the compromise between enthalpy (Intra- and intermolecular interactions) and entropic (related to the number of accessible states) contributions\cite{1}\cite{2}\cite{4}\cite{7}. The knowledge of the tertiary structure of a protein is usually of great importance in biology as it permits: the analysis and prediction of protein function in cells; the identification of active sites and binding sites of a receptor; or the identification of a site of recombination to the action of another protein\cite{2}. Finally, the quaternary structure is defined for proteins having multiple domains of tertiary structure and is defined by the spatial arrangement of these domains.

One of the main challenges in structural bioinformatics concerns the determination of protein tertiary structures. Determining the tertiary structure of proteins are experimentally and computationally expensive and time consuming\cite{8}. The difficulty in determining 3D structures of proteins has generated a large imbalance between the number of known amino acid sequences and the number of 3D structures of proteins. In spite of a large number of known protein sequences, only a small fraction possess associated 3D structures. This emphasizes the need of computational methods for the prediction of protein tertiary structures.


A number of computational methods have been proposed as a solution to the problem of protein structure prediction (PSP)\cite{9}\cite{10}\cite{11}\cite{12}. These methods can be divided into four classes \cite{13}: (a) fold recognition and threading methods\cite{14}\cite{15}\cite{16}\cite{17}; (b) comparative modeling methods and sequence alignment strategies \cite{18}\cite{19} (c) first principle methods with database information\cite{20}\cite{21}; (d) first principle methods without database information\cite{10}. The first three groups of methods are capable of predicting the tertiary structure of proteins quickly and efficiently when template structures or folding libraries are known\cite{22}. However, the last group (first principle methods without database information), which is referred to as ab initio, is based on first principles and do not rely on database information and produces new folds through computer simulation of physicochemical properties of the process of protein folding in nature. All these techniques generate a large number of candidate models known as decoys \cite{23}. Determine the models quality assessment is a challenging problem in structural biology and is an important area in bioinformatics\cite{24}.Several model quality assessment methods(MQA) have been developed since CASP7 (7th Community Wide Experiment on the Critical Assessment of Techniques for Protein Structure Prediction)\cite{24}\cite{25}.

Over the past years, several techniques have been developed, these techniques can be classified into two big groups: When the  native protein structure is known and when is not. The first group of techniques evaluates the quality of a decoy through different measures of similarity such as Root-Mean-Square Deviation (RMSD)\cite{26}, Global Distance Test Total Score (GDT\_TS) \cite{27}, Template-Modeling Score (TM-Score)\cite{28} and Maximal Substructure (MaxSub)\cite{29}. The second group of methods uses scoring functions that allows one to discriminate between high and low-quality models. These scoring functions can be classified as follows: (a)physics based potential functions; (b)statistical potential functions; (c)consensus-based functions; and (d)machine learning algorithms.Physics based functions calculate the potential energy associated with a given protein model and may incorporate also the interactions with the surrounding solvent\cite{30}\cite{31}. Such methods usually consume a significant amount of computational time and are not very sensitive to small atomic changes. Statistical potential functions evaluate the quality of the  models based on statistical analysis of structural attributes extracted from a known protein structure database\cite{32}\cite{33}. However, these methods only consider average properties of known protein structures and therefore have limited power to discriminate and rank structural models. Consensus-based functions have a good performance when many models in the database are similar to the native structure. However, if the database is composed of only low-quality models, these methods tend to show less performance than knowledge-based approaches\cite{34}\cite{35}. Machine learning algorithms, such as support vector machine (SVM), neural networks (NN) and random forest (RF), evaluate the quality of the models in relation to certain features or attributes\cite{36}\cite{37}. These features are extracted from the sequence and structure of decoy models and are used as inputs to evaluate their quality. The great advantage of these methods is that they can consider a large number of features. These features may be considered simultaneously, and the hidden relationships between them can be taken into account, which is very difficult to achieve with the other methods described.
% % %\newpage

Features can be extracted from the secondary structure (for example, average of amino acids forming alpha helix structures). Features based on Euclidean distance are also widely used (for example, distance between the amino acids and solvent distance between aliphatic amino acids). Features can also be calculated based on the physicochemical properties of protein models, such as the number of aliphatic hydrophobic amino acids and the solvent accessibility on different levels\cite{38}\cite{39}\cite{40}. Furthermore, it is also possible to represent the protein as a network structure\cite{41}. Using such a representation it is possible to calculate features such as: (a)number of non covalent interactions (defined by the number of sides on the network); (b)set of nodes connected with maximum number of amino acids; (c)average cluster coefficient of the network; and (d)average cluster coefficient of the largest cluster. 

Given this wide range of possible features, many studies select an initial set to be used as inputs for the machine learning model. This selection is performed in an arbitrary fashion, which implies in biased results and, therefore, limits the prediction power of the method.

Therefore, it is highly desirable to obtain an unbiased set of relevant features in an automatic way. This could be achieved by a computational method capable of, in a first step, perform automatic feature selection identifying the most important ones from a large pool of available features and, in a second step, evaluate the relative importance of each feature within this reduced subset.

\section{Objectives}
The main objective of this study is to calculate the importance of different types of features in protein quality assessment. The specific objectives are:

\begin{enumerate}[label=(\roman{*})]
  \item To propose a model that is able to evaluate the relative importance for an optimal subset of features;
  \item To use this model to predict quality of decoy models towards their native structures. 
\end{enumerate}

\section{Contributions}
The main contribution of this work is the proposal of a strategy that uses a hybrid model, based on a genetic algorithm and a neural network, for the calculation of the characteristics (features). This model allows specifically:
\begin{enumerate}[label=(\roman{*})]
	\item Calculate three different scores for each of the features used in this proposal;
	\item Calculate the similarity of decoys models relative to its native structure. 
\end{enumerate}

\section{Work Description}
The development of this study was done in the following steps:
\begin{enumerate}[label=(\roman{*})]
	\item Study of the main technical similarity in the literature such as RMSD, TM-score, GDT\_TS, MaxSub.
	\item Study of machine learning techniques to solve this problem (Neural Networks, SVMR, Random Forest etc.). 
	\item Search of features commonly used in machine learning models described above
	\item Pre-processing decoys database, which is necessary for the feature extraction (calculation). 
	\item Hybrid model definition (GA + NN), for the calculation of features importance.
	\item Implementation (Matlab and Python) and testing of the proposed model in programming languages, to adjust the proposed model.
\end{enumerate}

\section{Work Organization}

The proposal is divided into four additional chapters. Chapter 2 presents a summary of the theoretical foundations necessary for understanding this work, which will consider issues such as amino acids, proteins and its different levels of organization, methods for predicting the 3D protein structure, protein structure similarity metrics, and machine learning methods to predict the quality of decoys. Finally, the chapter shows the different features used to predict the protein quality assessment. In Chapter 3, the model for the protein quality assessment calculation will be presented in details, emphasizing the calculation mechanisms of the importance of the features. The chapter 4 describes and analyzes the preliminary results obtained until now. Finally, Chapter 5 presents the discussions and conclusions on the proposed model and the results obtained. 
  \input{chapter-2}
  \input{conclusions}
  %% ...
  \arial
  \bibliography{edwin}
  \normalfont
  \input{appendix}
\end{document}
\grid
